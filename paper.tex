\documentclass[conference]{IEEEtran}
\IEEEoverridecommandlockouts
\usepackage{cite}
\usepackage{amsmath,amssymb,amsfonts}
\usepackage{algorithmic}
\usepackage{graphicx}
\usepackage{textcomp}
\usepackage{xcolor}
\def\BibTeX{{\rm B\kern-.05em{\sc i\kern-.025em b}\kern-.08em
    T\kern-.1667em\lower.7ex\hbox{E}\kern-.125emX}}
\begin{document}

\title{Autonomous Firefighting Drone}

\author{
\IEEEauthorblockN{
Ashvith Shetty,
Aditya Rao Chokkadi,
Atul Singh Parmar and
Adithya Shet
}
\IEEEauthorblockA{Department of Computer Science Engineering\\
NMAM Institute of Technology, Udupi, Karnataka}
}

\maketitle

\begin{abstract}
This work showcases an autonomous drone that can take part in rescue missions. The scope of the project is limited to fire detection and extinguishment. We have used a custom-data trained YOLOv5s for flame detection. YOLOv5 is a family of object detection models developed by Ultralytics.
\end{abstract}

\begin{IEEEkeywords}
autonomous, drone, yolov5, flame-detection, deep-learning, computer-vision, real-time, cnn, video-fire-detection
\end{IEEEkeywords}

\section{Introduction}
This is our introduction

\section{Literature Review}
%Although there were several controversy around the YOLOv5 model underperforming the YOLOv4 model, the sheer amount of well-written blogs and documentation was what made us go with YOLOv5.


\section{Methodology}
This is where we write methodology

\subsection{Methodology-1}\label{AA}
This is a subsection of methodology 

\subsection{Methodology-2}
\begin{itemize}
\item This is an item.
\end{itemize}

\subsection{Methodology-3}
Write equation here:
\begin{equation}
a+b=\gamma\label{eq}
\end{equation}

\subsection{Methodology-4}
\paragraph{Subsection inside methodology, but with alphabetical bullets} Something to write about.
\begin{table}[htbp]
\caption{Table Type Styles}
\begin{center}
\begin{tabular}{|c|c|c|c|}
\hline
\textbf{Table}&\multicolumn{3}{|c|}{\textbf{Table Column Head}} \\
\cline{2-4} 
\textbf{Head} & \textbf{\textit{Table column subhead}}& \textbf{\textit{Subhead}}& \textbf{\textit{Subhead}} \\
\hline
copy& More table copy$^{\mathrm{a}}$& &  \\
\hline
\multicolumn{4}{l}{$^{\mathrm{a}}$Sample of a Table footnote.}
\end{tabular}
\label{tab1}
\end{center}
\end{table}

\begin{figure}[htbp]
\centerline{
%includegraphics{}
}
\caption{Example of a figure caption.}
\label{fig}
\end{figure}

\section*{Result}

This is our result.

\section*{Implementation}

This is our acknowledgement.

\section*{Conclusion}

This is our conclusion


\begin{thebibliography}{00}
\bibitem{b1} Ahmed Saied, Ahmed Gamaleldin, Ahmed Atef, Heba Saker and Ahmed Shaheen (2018), ``FIRE dataset, Version 1''. Retrieved March 02, 2022 from https://www.kaggle.com/datasets/phylake1337/fire-dataset.

\bibitem{b2} Moses Olafenwa (2019), ``FireNet dataset, Version 1". Retrieved April 05, 2022 from https://github.com/OlafenwaMoses/FireNET.

\bibitem{b3} Snehit Vaddi (2021), ``YOLOv3 Fire Detection dataset, Version 1''. Retrieved April 05, 2022 from https://github.com/snehitvaddi/YOLOv3-Cloud-Based-Fire-Detection.

\bibitem{b4} Kutay Kutlu (2021), ``Forest Fire, Version 1''. Retrieved April 07, 2022 from https://www.kaggle.com/datasets/kutaykutlu/forest-fire.

%\bibitem{b1} G. Eason, B. Noble, and I. N. Sneddon, ``On certain integrals of Lipschitz-Hankel type involving products of Bessel functions,'' Phil. Trans. Roy. Soc. London, vol. A247, pp. 529--551, April 1955.
\end{thebibliography}
\end{document}
