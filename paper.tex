\documentclass[conference]{IEEEtran}
\IEEEoverridecommandlockouts
% The preceding line is only needed to identify funding in the first footnote. If that is unneeded, please comment it out.
\usepackage{cite}
\usepackage{amsmath,amssymb,amsfonts}
\usepackage{algorithmic}
\usepackage{graphicx}
\usepackage{textcomp}
\usepackage{xcolor}
\def\BibTeX{{\rm B\kern-.05em{\sc i\kern-.025em b}\kern-.08em
    T\kern-.1667em\lower.7ex\hbox{E}\kern-.125emX}}
\begin{document}

\title{Autonomous Firefighting Drone}

\author{
\IEEEauthorblockN{
Ashvith Shetty,
Aditya Rao Chokkadi,
Atul Singh Parmar and
Adithya Shet
}
\IEEEauthorblockA{Department of Computer Science Engineering\\
NMAM Institute of Technology, Udupi, Karnataka}
}

\maketitle

\begin{abstract}
This work showcases an implementation of an autonomous drone that can take part in rescue missions. The scope of the project is limited to fire detection and extinguishment. We have used a custom-data trained YOLOv5s for flame detection. YOLOv5 is the latest object detection model developed by ultralytics. %Although there was a controversy around the YOLOv5 model being worse than YOLOv5, it was faster to train than the other 
\end{abstract}

\begin{IEEEkeywords}
autonomous, drone, yolo, flame-detection, deep-learning, computer-vision
\end{IEEEkeywords}

\section{Introduction}
This is our introduction

\section{Literature Review}

\subsection{Option A}

Something about Option A.

\section{Methodology}
This is where we write methodology

\subsection{Methodology-1}\label{AA}
This is a subsection of methodology 

\subsection{Methodology-2}
\begin{itemize}
\item This is an item.
\end{itemize}

\subsection{Methodology-3}
Write equation here:
\begin{equation}
a+b=\gamma\label{eq}
\end{equation}

\subsection{Methodology-4}
\paragraph{Subsection inside methodology, but with alphabetical bullets} Something to write about.
\begin{table}[htbp]
\caption{Table Type Styles}
\begin{center}
\begin{tabular}{|c|c|c|c|}
\hline
\textbf{Table}&\multicolumn{3}{|c|}{\textbf{Table Column Head}} \\
\cline{2-4} 
\textbf{Head} & \textbf{\textit{Table column subhead}}& \textbf{\textit{Subhead}}& \textbf{\textit{Subhead}} \\
\hline
copy& More table copy$^{\mathrm{a}}$& &  \\
\hline
\multicolumn{4}{l}{$^{\mathrm{a}}$Sample of a Table footnote.}
\end{tabular}
\label{tab1}
\end{center}
\end{table}

\begin{figure}[htbp]
\centerline{
%includegraphics{}
}
\caption{Example of a figure caption.}
\label{fig}
\end{figure}

Figure Labels: Use 8 point Times New Roman for Figure labels. Use words 
rather than symbols or abbreviations when writing Figure axis labels to 
avoid confusing the reader. As an example, write the quantity 
``Magnetization'', or ``Magnetization, M'', not just ``M''. If including 
units in the label, present them within parentheses. Do not label axes only 
with units. In the example, write ``Magnetization (A/m)'' or ``Magnetization 
\{A[m(1)]\}'', not just ``A/m''. Do not label axes with a ratio of 
quantities and units. For example, write ``Temperature (K)'', not 
``Temperature/K''.

\section*{Acknowledgment}

This is our acknowledgement.

\section*{References}

This is references - we can cite too: \cite{b1}.

\begin{thebibliography}{00}
\bibitem{b1} G. Eason, B. Noble, and I. N. Sneddon, ``On certain integrals of Lipschitz-Hankel type involving products of Bessel functions,'' Phil. Trans. Roy. Soc. London, vol. A247, pp. 529--551, April 1955.
\end{thebibliography}
\end{document}
