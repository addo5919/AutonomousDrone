\documentclass[conference]{IEEEtran}
\IEEEoverridecommandlockouts
% The preceding line is only needed to identify funding in the first footnote. If that is unneeded, please comment it out.
\usepackage{cite}
\usepackage{amsmath,amssymb,amsfonts}
\usepackage{algorithmic}
\usepackage{graphicx}
\usepackage{textcomp}
\usepackage{xcolor}
\def\BibTeX{{\rm B\kern-.05em{\sc i\kern-.025em b}\kern-.08em
    T\kern-.1667em\lower.7ex\hbox{E}\kern-.125emX}}
\begin{document}

\title{Autonomous Firefighting Drone}

\author{
\IEEEauthorblockN{
Ashvith Shetty,
Aditya Rao Chokkadi,
Atul Singh Parmar and
Adithya Shet
}
\IEEEauthorblockA{Department of Computer Science Engineering\\
NMAM Institute of Technology, Udupi, Karnataka}
}

\maketitle

\begin{abstract}
This work showcases an implementation of an autonomous drone that can take part in rescue missions. The scope of the project is limited to fire detection and extinguishment. We have used a custom-data trained YOLOv5s for flame detection. YOLOv5 is the latest object detection model developed by ultralytics. %Although there was a controversy around the YOLOv5 model being worse than YOLOv5, it was faster to train than the other 
\end{abstract}

\begin{IEEEkeywords}
autonomous, drone, yolo, flame-detection, deep-learning, computer-vision
\end{IEEEkeywords}

\section{Introduction}
As a fire erupts[1], the first few minutes can be critical, and first respondents must race to the scene to analyze the situation and act fast before it gets out of hand. Factors such as road traffic condition and distance may not allow quick rescue operation using traditional means and methods, leading to unmanageable spreading of fire, injuries or even deaths that can be avoided. The issue with combating fire stems from one critical point, time. As a fire ignites, the flames may
spread quickly, and those first few moments can be quite critical. Firefighting tools have always been low-
tech such as trucks, ladders and hoses which does not always allow for quick extinguish time. Having a
delayed response can further worsen the situation and sometimes lead to uncontrollable state of affairs.
Another critical hurdle consists in the skyscrapers’ heights at which some of these fires may erupt. With conventional firefighting equipment it can be impossible to reach high story buildings and therefore any
attempts to curb the fire will involve fire fighters be futile. Furthermore, the lack of information at the fire
scene can lead to firefighters’ inability to containing the fire. Information such as how extensive the fire has
spread and how many potential victims may be trapped. All that plays an important role into the issues that
are dealt with when a fire erupts. One of the solutions to this problem is to deploy a squad of drones to the fire site. The drones will be equipped with fire detection capabilities (YOLOv5) and will be able to extinguish the fires via the attached extinguisher module.  The integration of drones with firefighting will allow for ease of access and control of fire outbreaks. Drones will also result in increased response time, prevention of further damage, and allow relaying of vital information to out of reach places regarding the characteristics of the fire scene.

\section{Literature Review}

\subsection{Option A}

Something about Option A.

\section{Methodology}


\subsection{Collecting Data}\label{AA}
\begin{itemize}
\item Fire Dataset- (https://www.kaggle.com/datasets/phylake1337/fire-dataset/metadata)
Authors of the dataset: Ahmed Saied,Ahmed Gamaleldin,Ahmed Atef,Heba Saker and Ahmed Shaheen.
\item FireNet dataset - (https://github.com/OlafenwaMoses/FireNET) Author of the dataset: Moses Olafenwa.
\item YOLOV3 Fire Detection dataset- (https://github.com/snehitvaddi/YOLOv3-Cloud-Based-Fire-Detection) Author of the dataset: Snehit Vaddi.
\item Forest Fire dataset- (https://www.kaggle.com/datasets/kutaykutlu/forest-fire) Author of the dataset: Kutay Kutlu.
\item Human Detection Dataset- (https://www.kaggle.com/datasets/constantinwerner/human-detection-dataset) Author of the dataset: Constantin Werner
\item LED light dataset- Web scraping
\end{itemize}


\subsection{Preparing the Data}
Roboflow website (https://roboflow.com/) was used for labelling the dataset. The dataset consists of 3,333 images out of which 1,033 images are negative samples. Negative samples consist of non-flame images: human images,smoke images and led light images. Roboflow's augmentation feature was made use of in order to generate more images for efficient training. Various versions of the dataset was generated in roboflow until the YOLOv5 model was accurately able to detect fire. The final version consists of 7991 images.


\subsection{Choosing a Model}
YOLOv5 model was chosen for flame detection because of the following reasons:-
\begin{itemize}
\item It is about 88 per cent smaller than YOLOv4.
\item It is about 180 per cent faster than YOLOv4.
\item It is roughly as accurate as YOLOv4 on the same task.
\end{itemize}


\subsection{Training the Model}
 In order to train the YOLOv5 model (available here-https://github.com/ultralytics/yolov5) the following steps are performed:-
 \begin{itemize}
 \item The github repository is cloned, all dependencies within requirements.txt are installed.
 \item YOLOv5n,YOLOv5s,YOLOv5m,YOLOv5l and YOLOv5x are the different pretrained models that YOLOv5 offers. YOLOv5s was chosen by us as it is the smallest and fastest model available.
 \item The final version of the dataset is extracted from Roboflow via code that it provides. The YOLOv5s model is trained on the Roboflow dataset by specifying dataset,batch-size,image size and pretrained weight(yolov5s.pt). Pretrained weights are auto-downloaded from the latest YOLOv5 release.
 \item The training results are saved to runs/train/ with incrementing run directories, i.e. runs/train/exp2, runs/train/exp3 etc.
\end{itemize} 


\subsection{Evaluating the Model}
\begin{itemize}
\item sample
\end{itemize}

\subsection{Parameter Tuning}
 We did not do any parameter tuning as it was not applicable for our case.
 
 \subsection{Making Predictions}
  After the YOLOv5 model has been trained with the Roboflow dataset, it can be used for fire detection. If an image containing a flame is given as input to the trained model then it will generate bounding boxes on the regions of the image where flames are detected.
 
 


\section*{Result}

This is our result.

\section*{Implementation}

This is our acknowledgement.

\section*{Conclusion}

This is our conclusion


\begin{thebibliography}{00}
\bibitem{b1} Hajer Ben Mnaouer, Mohammad Faieq, Adel Yousefi, Sarra Ben Mnaouer, ``FireFly: Autonomous Rescue Drones`` Canadian University Dubai, First Interchange Sheikh Zaid Road, Dubai, United Arab Emirates.
\end{thebibliography}
\end{document}
