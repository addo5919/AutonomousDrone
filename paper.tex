\documentclass[conference]{IEEEtran}
\IEEEoverridecommandlockouts
% The preceding line is only needed to identify funding in the first footnote. If that is unneeded, please comment it out.
\usepackage{cite}
\usepackage{amsmath,amssymb,amsfonts}
\usepackage{algorithmic}
\usepackage{graphicx}
\usepackage{textcomp}
\usepackage{xcolor}
\def\BibTeX{{\rm B\kern-.05em{\sc i\kern-.025em b}\kern-.08em
    T\kern-.1667em\lower.7ex\hbox{E}\kern-.125emX}}
\begin{document}

\title{Autonomous Firefighting Drone*\\
%{\footnotesize \textsuperscript{*}Note: Sub-titles are not captured in Xplore and
%should not be used}
%\thanks{Identify applicable funding agency here. If none, delete this.}
}

\author{\IEEEauthorblockN{Ashvith Shetty}
\IEEEauthorblockA{\textit{Computer Science and Engineering} \\
\textit{NMAM Institute of Technology}\\
Udupi, Karnataka \\
<email address or ORCID>}
\and
\IEEEauthorblockN{Aditya Rao Chokkadi}
\IEEEauthorblockA{\textit{Computer Science and Engineering} \\
\textit{NMAM Institute of Technology}\\
Udupi, Karnataka}
\and
\IEEEauthorblockN{Atul Singh Parmar}
\IEEEauthorblockA{\textit{Computer Science and Engineering} \\
\textit{NMAM Institute of Technology}\\
Udupi, Karnataka}
\and
\IEEEauthorblockN{Adithya Shet}
\IEEEauthorblockA{\textit{Computer Science and Engineering} \\
\textit{NMAM Institute of Technology}\\
Udupi, Karnataka}
}

\maketitle

\begin{abstract}
This is our sample abstract.
\end{abstract}

\begin{IEEEkeywords}
autonomous, drone, yolo, flame-detection, deep-learning, computer-vision
\end{IEEEkeywords}

\section{Introduction}
This is our introduction

\section{Literature Review}

\subsection{Option A}

Something about Option A.

\section{Methodology}
This is where we write methodology

\subsection{Methodology-1}\label{AA}
This is a subsection of methodology 

\subsection{Methodology-2}
\begin{itemize}
\item This is an item.
\end{itemize}

\subsection{Methodology-3}
Write equation here:
\begin{equation}
a+b=\gamma\label{eq}
\end{equation}

\subsection{Figures and Tables}
\paragraph{Positioning Figures and Tables} Place figures and tables at the top and 
bottom of columns. Avoid placing them in the middle of columns. Large 
figures and tables may span across both columns. Figure captions should be 
below the figures; table heads should appear above the tables. Insert 
figures and tables after they are cited in the text. Use the abbreviation 
``Fig.~\ref{fig}'', even at the beginning of a sentence.

\begin{table}[htbp]
\caption{Table Type Styles}
\begin{center}
\begin{tabular}{|c|c|c|c|}
\hline
\textbf{Table}&\multicolumn{3}{|c|}{\textbf{Table Column Head}} \\
\cline{2-4} 
\textbf{Head} & \textbf{\textit{Table column subhead}}& \textbf{\textit{Subhead}}& \textbf{\textit{Subhead}} \\
\hline
copy& More table copy$^{\mathrm{a}}$& &  \\
\hline
\multicolumn{4}{l}{$^{\mathrm{a}}$Sample of a Table footnote.}
\end{tabular}
\label{tab1}
\end{center}
\end{table}

\begin{figure}[htbp]
\centerline{\includegraphics{fig1.png}}
\caption{Example of a figure caption.}
\label{fig}
\end{figure}

Figure Labels: Use 8 point Times New Roman for Figure labels. Use words 
rather than symbols or abbreviations when writing Figure axis labels to 
avoid confusing the reader. As an example, write the quantity 
``Magnetization'', or ``Magnetization, M'', not just ``M''. If including 
units in the label, present them within parentheses. Do not label axes only 
with units. In the example, write ``Magnetization (A/m)'' or ``Magnetization 
\{A[m(1)]\}'', not just ``A/m''. Do not label axes with a ratio of 
quantities and units. For example, write ``Temperature (K)'', not 
``Temperature/K''.

\section*{Acknowledgment}

The preferred spelling of the word ``acknowledgment'' in America is without 
an ``e'' after the ``g''. Avoid the stilted expression ``one of us (R. B. 
G.) thanks $\ldots$''. Instead, try ``R. B. G. thanks$\ldots$''. Put sponsor 
acknowledgments in the unnumbered footnote on the first page.

\section*{References}
Please number citations consecutively within brackets \cite{b1}.

\begin{thebibliography}{00}
\bibitem{b1} G. Eason, B. Noble, and I. N. Sneddon, ``On certain integrals of Lipschitz-Hankel type involving products of Bessel functions,'' Phil. Trans. Roy. Soc. London, vol. A247, pp. 529--551, April 1955.
\end{thebibliography}
\end{document}
