\documentclass[conference]{IEEEtran}
\IEEEoverridecommandlockouts
% The preceding line is only needed to identify funding in the first footnote. If that is unneeded, please comment it out.
\usepackage{cite}
\usepackage{amsmath,amssymb,amsfonts}
\usepackage{algorithmic}
\usepackage{graphicx}
\usepackage{textcomp}
\usepackage{xcolor}
\def\BibTeX{{\rm B\kern-.05em{\sc i\kern-.025em b}\kern-.08em
    T\kern-.1667em\lower.7ex\hbox{E}\kern-.125emX}}
\begin{document}

\title{Autonomous Firefighting Drone}

\author{
\IEEEauthorblockN{
Ashvith Shetty,
Aditya Rao Chokkadi,
Atul Singh Parmar and
Adithya Shet
}
\IEEEauthorblockA{Department of Computer Science Engineering\\
NMAM Institute of Technology, Udupi, Karnataka}
}

\maketitle

\begin{abstract}
This work showcases an implementation of an autonomous drone that can take part in rescue missions. The scope of the project is limited to fire detection and extinguishment. We have used a custom-data trained YOLOv5s for flame detection. YOLOv5 is the latest object detection model developed by ultralytics. %Although there was a controversy around the YOLOv5 model being worse than YOLOv5, it was faster to train than the other 
\end{abstract}

\begin{IEEEkeywords}
autonomous, drone, yolo, flame-detection, deep-learning, computer-vision
\end{IEEEkeywords}

\section{Introduction}
As a fire erupts[2], the first few minutes can be critical, and first respondents must race to the scene to analyze the situation and act fast before it gets out of hand. Factors such as road traffic condition and distance may not allow quick rescue operation using traditional means and methods, leading to unmanageable spreading of fire, injuries or even deaths that can be avoided. The issue with combating fire stems from one critical point, time. As a fire ignites, the flames may
spread quickly, and those first few moments can be quite critical. Firefighting tools have always been low-
tech such as trucks, ladders and hoses which does not always allow for quick extinguish time. Having a
delayed response can further worsen the situation and sometimes lead to uncontrollable state of affairs.
Another critical hurdle consists in the skyscrapers’ heights at which some of these fires may erupt. With conventional firefighting equipment it can be impossible to reach high story buildings and therefore any
attempts to curb the fire will involve fire fighters be futile. Furthermore, the lack of information at the fire
scene can lead to firefighters’ inability to containing the fire. Information such as how extensive the fire has
spread and how many potential victims may be trapped. All that plays an important role into the issues that
are dealt with when a fire erupts. One of the solutions to this problem is to deploy a squad of drones to the fire site. The drones will be equipped with fire detection capabilities (YOLOv5) and will be able to extinguish the fires via the attached extinguisher module.  The integration of drones with firefighting will allow for ease of access and control of fire outbreaks. Drones will also result in increased response time, prevention of further damage, and allow relaying of vital information to out of reach places regarding the characteristics of the fire scene.

\section{Literature Review}

\subsection{Option A}

Something about Option A.

\section{Methodology}
This is where we write methodology

\subsection{Methodology-1}\label{AA}
This is a subsection of methodology 

\subsection{Methodology-2}
\begin{itemize}
\item This is an item.
\end{itemize}

\subsection{Methodology-3}
Write equation here:
\begin{equation}
a+b=\gamma\label{eq}
\end{equation}

\subsection{Methodology-4}
\paragraph{Subsection inside methodology, but with alphabetical bullets} Something to write about.
\begin{table}[htbp]
\caption{Table Type Styles}
\begin{center}
\begin{tabular}{|c|c|c|c|}
\hline
\textbf{Table}&\multicolumn{3}{|c|}{\textbf{Table Column Head}} \\
\cline{2-4} 
\textbf{Head} & \textbf{\textit{Table column subhead}}& \textbf{\textit{Subhead}}& \textbf{\textit{Subhead}} \\
\hline
copy& More table copy$^{\mathrm{a}}$& &  \\
\hline
\multicolumn{4}{l}{$^{\mathrm{a}}$Sample of a Table footnote.}
\end{tabular}
\label{tab1}
\end{center}
\end{table}

\begin{figure}[htbp]
\centerline{
%includegraphics{}
}
\caption{Example of a figure caption.}
\label{fig}
\end{figure}

\section*{Result}

This is our result.

\section*{Implementation}

This is our acknowledgement.

\section*{Conclusion}

This is our conclusion


\begin{thebibliography}{00}
\bibitem{b1} G. Eason, B. Noble, and I. N. Sneddon, ``On certain integrals of Lipschitz-Hankel type involving products of Bessel functions,'' Phil. Trans. Roy. Soc. London, vol. A247, pp. 529--551, April 1955.
\bibitem{b2} Hajer Ben Mnaouer, Mohammad Faieq, Adel Yousefi, Sarra Ben Mnaouer, ``FireFly: Autonomous Rescue Drones`` Canadian University Dubai, First Interchange Sheikh Zaid Road, Dubai, United Arab Emirates
\end{thebibliography}
\end{document}
